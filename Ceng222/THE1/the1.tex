\documentclass[12pt]{article}
\usepackage[utf8]{inputenc}
\usepackage{float}
\usepackage{amsmath}


\usepackage[hmargin=3cm,vmargin=6.0cm]{geometry}
%\topmargin=0cm
\topmargin=-2cm
\addtolength{\textheight}{6.5cm}
\addtolength{\textwidth}{2.0cm}
%\setlength{\leftmargin}{-5cm}
\setlength{\oddsidemargin}{0.0cm}
\setlength{\evensidemargin}{0.0cm}

%misc libraries goes here
\usepackage{tikz}
\usetikzlibrary{automata,positioning}

\begin{document}

\section*{Student Information } 
%Write your full name and id number between the colon and newline
%Put one empty space character after colon and before newline
Full Name : Furkan Göksel \\
Id Number : 2237436 \\

% Write your answers below the section tags
\section*{Answer 1}

\subsection*{a.}
$P(Z=0 \; | \; X=5, Y=10 ) = \dfrac{P(Z=0 \cap X =5,Y=10)}{P(X=5,Y=10)}$ \\ \\
In order to calculate $P(Z=0 \; \cap \; X =5,Y=10)$ from joint distribution we calculate $P_{(x,y,z)}(5,10,0) = 0.075$ and to find $P(X=5,Y=10)$ we have to calculate $\sum_{z}P_{(x,y,z)}(5,10,z) = 0.075+0.050 = 0.125$ so the result is $\frac{0.075}{0.125} = 0.6$
\subsection*{b.}
To find marginal distribution of P(X = x) we have to calculate (from addition rule) $\sum_{y}\sum_{z}P_{(x,y,z)}(x,y,z)$
\begin{equation} 
\begin{split}
P(X=3) = 0.025+0.025+0.03+0.02+0.05+0.15 = 0.3 \\
P(X=5) = 0.075+0.050+0.025+0.030+0.020+0.2 = 0.4 \\
P(X=7) = 0.04+0.06+0.025+0.050+0.025+0.1 = 0.3 \\
\end{split}
\end{equation}
To verify let's check $\sum_xP_x(x) = 0.3+0.4+0.3=1$
\subsection*{c.}
$E_X[X]= \sum_x(xP(x))$, so $E_X[x] = 3*0.3+5*0.4+7*0.3 = 5$ \\
$Var_X[X] = \sum_x(x-\mu)^2P(x) = (3-5)^2*0.3+(5-5)^2*0.4+(7-5)^2*0.3 =2.4$
\subsection*{d.}
From the Joint distribution and conditional probability rules, it is given the fact that Z = 1 so, we just calculate
$\dfrac{P(x \cap Z=1)}{P(Z=1)}$ \\
$P(Z=1)=\sum_{y}P_{(x,y,z)}(x,y,1) = 0.025+0.050+0.06+0.02+0.03+0.05+0.15+0.2+0.1=0.685$ \\
For marginal distribution of  $P(X\cap Z=1)$ $\sum_{y}P_{(x,y,z)}(x,y,1)$:
\begin{equation} 
\begin{split}
P(X=3) = 0.025+0.02+0.15 = 0.195 \\
P(X=5) = 0.050+0.030+0.2 = 0.280 \\
P(X=7) = 0.06+0.050+0.1 = 0.210 \\
\end{split}
\end{equation}
Hence ; \\
$ P(X=3 | Z=1) = \dfrac{0.195}{0.685}= 0.285$ \\ \\
$ P(X=5 | Z=1) = \dfrac{0.280}{0.685} = 0.409$ \\ \\
$ P(X=7 | Z=1) = \dfrac{0.210}{0.685} = 0.306$ \\ \\
\subsection*{e.}
$E_{[X | Z=1]}= \sum_x(xP(x \; | \; Z =1)) = 3*0.285+5*0.409+7*0.306 = 5.042$ \\
$Var_{[X | Z=1]} = \sum_x(x-\mu)^2P(x \; | \; Z =1) = (3-5.042)^2*0.285+(5-5.042)^2*0.409+(7-5.042)^2*0.306 = 2.3622$
\section*{Answer 2}

\subsection*{a.}
We know that cumulative distribution of these converge to 1. So we can say that \\
$F_x(\infty) = c_1\sum_{k=1}^{\infty}(2^{-k}) = 1 $ then we know that from geometric series test $\sum_{k=1}^{\infty}(2^{-k}) = 1$ so $c_1 = 1$ \\
$F_y(\infty = c_2\sum_{k=1}^{\infty}(\dfrac{2^{-k}}{k}) = 1$ then by the ratio test, the series converges to $log2$ and $c_2 = 1/log2 = 1.4427$
\subsection*{b.}
Since $c_1 = 1$, P(X is even)$= P(2)+P(4)+P(6)+P(8)+..= \sum_{k=1}^{\infty}(2^{-2k}) =\dfrac{1}{3}$
\subsection*{c.}
From Bayes Formula $P(X+Y=6 | $ X is odd) = $\dfrac{P(X  \; \;is\; \; odd | \; X+Y=6 )P(X+Y=6)}{P(X \; \; is \; \;odd )}$ \\
P(X is odd) = 1- P(X is even) = $\dfrac{2}{3}$ \\
P(X+Y=6) = $P_X(1)P_Y(5)+P_X(2)P_Y(4)+P_X(3)P_Y(3)+P_X(4)P_Y(2)+P_X(5)P_Y(1) = 0.051$ \\ \\
$P(X \; \;is\; \; odd \; | \; X+Y=6 ) = \dfrac{P(X \; \;is\; \; odd \; \cap X+Y=6 )}{P(X+Y=6)}$ = $\dfrac{0.345}{0.051}=0.676$ \\ \\
$\dfrac{P(X  \; \;is\; \; odd | \; X+Y=6 )P(X+Y=6)}{P(X \; \; is \; \;odd )} = \dfrac{0.676*0.051}{2/3} = 0.051$
\subsection*{d.}
Choosing two ball is ${1000}\choose{2}$ = 499500. \\
Then for $b_1+b_2=6$, our probabilities are $P_x(1)P_y(5)+P_x(2)P_y(4)+P_x(4)P_Y(2)+P_x(5)P_Y(1)=0.044$ (There is no 3-3 possibility because balls are unique.) 
Then our result is $\dfrac{0.044}{499500}=0.000000088$ (from bayes).
\section*{Answer 3}

\subsection*{a.}
Random variables X and Y are independent if $P_{(X,Y)}(x,y) = P_X(x)P_Y(y)$ for all values of x and y. This means, events \{X = x\} and \{Y = y\} are independent for all x and y. So just one counter example is enough. When we looked the $P_{(X,Y)}(3,10) = 0.05$ and from their joint distributions, $P_Y(Y=10) = 0.025+0.025+0.075+0.050+0.04+0.06 = 0.275$ and $P_X(X=3) = 0.3$ (from question 1) and $0.3*0.275 = 0.0825 \neq 0.05$, hence they are not independent.
\subsection*{b.}
If event A independent of itself, we can say that \\
\begin{gather*} 
P(A \cap A') = P(A)P(A')
\end{gather*}
from there we know that $P(A \cap A') = 0$ then our equation becomes 
\begin{gather*}
P(A)P(A') = 0
\end{gather*}
Since $P(A') = 1 - P(A)$ our equation equals to
\begin{gather*}
P(A)(1-P(A)) = 0
\end{gather*}
in order to get 0, P(A) must be equal either 1 or 0. Hence we proved that if event A is  independent of itself, then P(A) is either 1 or 0. (P(A') means complement of P(A)) 
\subsection*{c.}
If two events are independent then $P(A\cap B) = P(A)P(B)$ holds. \\
If P(A) = 0 we can say that $P(A\cap B) = 0$ and then $P(A \cap B) = 0 = P(A) = P(A)P(B)$ \\
If P(A) = 1 then P(A') = 0 (P(A') = 1-P(A)). So we can easily say that $P(A\cap B) = P(A \cap B) + P(A' \cap B)$ (since $P(A' \cap B)$ is 0 nothing changes) and then it is equal to :
\begin{gather*}
P((A \cap B) \cup (A' \cap B)) = P((A\cup A')\cap B) \\
=P(1 \cap B) = P(B) \\
=P(B).1 = P(B).P(A)
\end{gather*}
Hence $P(A \cap B) = P(A)P(B)$. So we have proved that If event A have P(A) = 0 or 1, then A and an arbitrary event B are independent.
\section*{Answer 4}

\subsection*{a.}
From conditional probability: 
\begin{gather*}
P(G = n-1+m | G > n-1 ) = \dfrac{P((G = n-1+m )\cap (G > n-1))}{P(G > n-1)} \\ \\
P((G = n-1+m )\cap (G > n-1)) = P(G = n-1+m) = p(1-p)^{n+m-2} \\
P(G > n-1) = P(n) + P(n+1) + P(n+2).. = p(1-p)^{n-1}+p(1-p)^{n}.. \\
=  p(1-p)^{n-1}(1+(1-p)+(1-p)^2..) = p(1-p)^{n-1}\dfrac{1}{1-(1-p)}=(1-p)^{n-1} \\
\end{gather*}
Hence;
\begin{gather*}
P(G = n-1+m | G > n-1 ) = \dfrac{p(1-p)^{n+m-2}}{(1-p)^{n-1}} =p(1-p)^{m-1} = P(G=m)
\end{gather*}
\subsection*{b.}
$F(n) = P(X \leq n)$ we can say that $P(X \leq n) = 1 - P(X>n)$ and it is equal to $1-\sum_{w=n+1}^\infty(p.(1-p)^{w-1}$ and then
\begin{gather*} 
F(n)= 1-\sum_{w=n+1}^\infty(p.(1-p)^{w-1} \\
    =1-p.(1-p)^n(1+(1-p)+(1-p)^2+..) \\
    =1-p(1-p)^n[\dfrac{1}{1-(1-p)}] \\
    =1-(1-p)^n
\end{gather*}
Hence $F(n) = 1-(1-p)^n$
\subsection*{c.}
In order to get 4 of sum of dices we have to get 112,121,211 and it is equal to $\dfrac{3}{6.6.6} = \dfrac{1}{72}$ then
\begin{gather*} 
P(65 \leq G \leq 75) = F(75) - F(64) \\
=(1-(1-\dfrac{1}{72})^{75})-(1-(1-\dfrac{1}{72})^{64}) = 0.0582
\end{gather*}
\section*{Answer 5}

\subsection*{a.}
In binomial distribution, we can think this problem like 20,000 try and exactly 3 success. And from given information we know that P(Mutation)=$\dfrac{1}{10,000}$ ,so from our formula we get that $\binom{20,000}{3}.(\dfrac{1}{10,000})^3.(1-\dfrac{1}{10,000})^{20,000-3}=0.1804$
\subsection*{b.}
I chose the poisson distribution, because binomial distribution can be approximated to poisson distribution when probability of success is small and number of trials is large and (n $\geq 30, p \leq 0.05$) get parameter $\lambda = np$ so $\lambda = \dfrac{1}{10000}*20000 = 2$ and from poisson distribution 
\begin{gather*} 
\dfrac{e^{-\lambda } \lambda^k}{k!} = \dfrac{e^{-2}2^3}{3!} = 0.1804
\end{gather*}
We get same accurance because when number of trials are big, the result is more accurate.
\section*{Answer 6}

\subsection*{a.}

Since $P_O(k;\lambda) = \dfrac{e^{-\lambda} \lambda^k}{k!} $ For $E(O^n)$
\begin{equation} 
\begin{split}
E(O^n) = \sum_{k=0}^{\infty}k^nP(O=k) = \sum_{k=0}^{\infty}k^n\dfrac{e^{-\lambda} \lambda^k}{k!}= \sum_{k=1}^{\infty}k^{n-1}\dfrac{e^{-\lambda} \lambda^k}{(k-1)!} = \lambda \sum_{k=1}^{\infty}k^{n-1}\dfrac{e^{-\lambda} \lambda^{k-1}}{(k-1)!}
\end{split}
\end{equation}
Now let's shift our index to 0.
\begin{equation} 
\begin{split}
\lambda \sum_{k=0}^{\infty}(k+1)^{n-1}\dfrac{e^{-\lambda} \lambda^{k}}{k!}
\end{split}
\end{equation}
And this equation is exactly equal to $\lambda E[(O+1)^{n-1}]$
\subsection*{b.}
According to our formula, $E[0^3] = \lambda E[(O+1)^2]$. Now let's take square of it $\lambda E[O^2+2O+1]$ and we know that from definiton, $E[O^2+2O+1] = E[O^2]+2E[O]+1$. And let's calculate $E[0^2]$ and $E[0]$ from our formula:
\begin{equation} 
\begin{split}
E[0] = \lambda E[1^0] = \lambda \\
E[0^2] = \lambda E[(O+1)] = \lambda (E[O] + E[1]) = \lambda ( \lambda +1) 
\end{split}
\end{equation}
Hence $E[0^3]$ is equal to $\lambda((\lambda ^2 +\lambda)+2\lambda +1 ) = \lambda ^3 + 3\lambda ^2 +\lambda $

\end{document}
