\documentclass[12pt]{article}
\usepackage[utf8]{inputenc}
\usepackage{float}
\usepackage{amsmath}


\usepackage[hmargin=3cm,vmargin=6.0cm]{geometry}
%\topmargin=0cm
\topmargin=-2cm
\addtolength{\textheight}{6.5cm}
\addtolength{\textwidth}{2.0cm}
%\setlength{\leftmargin}{-5cm}
\setlength{\oddsidemargin}{0.0cm}
\setlength{\evensidemargin}{0.0cm}



\begin{document}

\section*{Student Information } 
%Write your full name and id number between the colon and newline
%Put one empty space character after colon and before newline
Full Name : Furkan Göksel \\
Id Number : 2237436 \\

% Write your answers below the section tags
\section*{Answer 1}
\subsection*{a} 
Green candies are in even numbers and less than five equal to this factor: $x^0+x^2+x^4$ \\
Red candies are more than three equal to this factor: $x^4+x^5+x^6+x^7+x^8...$ \\
Blue candies are in odd numbers equal to this factor: $x^1+x^3+x^5+x^7+x^9+....$ \\
And then we have to find coefficient of $x^{10}$ from this equation:
\begin{equation} 
\begin{split}
    (x^0+x^2+x^4)(x^4+x^5+x^6+x^7...)(x+x^3+x^5+x^7+x^9...) \\
    =(x^4+x^5+2x^6+2x^7+x3x^8+3x^9+3x^{10}+...)(x+x^3+x^5+x^7+x^9+...) \\
    =(...x^{10}+2x^{10}+3x^{10}...)
\end{split}    
\end{equation}
Then, from this equation, we can see that coefficients of $x^{10}$ is 6. Therefore, there are 6 ways to select 10 candies in such this way.
\subsection*{b}
When we assume having only five of each of types in hand, our factors become:\\
Green: $x^0+x^2+x^4$ \\
Red: $x^4+x^5$ \\
Blue: $x^1+x^3+x^5$ \\
And again we have to find coefficient of $x^{10}$ from this equation:
\begin{equation} 
\begin{split}
    (x^0+x^2+x^4)(x^4+x^5)(x+x^3+x^5) = (x^4+x^5+x^6+x^7+x^8+x^9)(x+x^3+x^5) \\
    =(...+x^{10}+x^{10}+x^{10}+..)
\end{split}    
\end{equation}
Then, from this equation, we can see that coefficients of $x^{10}$ is 3. Therefore, there are 3 ways to select 10 candies in such this way.
\subsection*{c}
We can say that:
\begin{equation} 
\begin{split}
    F(x) = \frac{x(7x)}{(1-2x)(1+3x)} \implies x(\frac{A}{1-2x} + \frac{B}{1+3x})=\frac{x(7x)}{(1-2x)(1+3x)} \\
    \implies \frac{A}{1-2x}+\frac{B}{1+3x} = \frac{7x}{(1-2x)(1+3x)}\implies A+3Ax+B-2Bx=7x \\ 
    \implies A=-B, 3A-2B=7 \implies A=\frac{7}{5}, B= \frac{-7}{5}
\end{split}    
\end{equation}
Also, we know that: \\
$\frac{1}{1-2x} \iff (2^0,2^1,2^2,2^3,2^4,2^5,....,2^n,...)$ \\
$\frac{1}{1+3x} \iff (3^0(-1)^0,3^1(-1)^1,3^2(-1)^2,....,3^n(-1)^n,....)$ \\
Then using these two equations we construct these sequence:
\begin{equation} 
\begin{split}
   \frac{1}{1-2x} - \frac{1}{1+3x} \iff (2^0-3^0(-1)^0,2^1-3^1(-1)^1,...,2^n-3^n(-1)^n,...) \\
   \frac{7}{5}(\frac{1}{1-2x} - \frac{1}{1+3x}) \iff (\frac{7}{5}(2^0-3^0(-1)^0),\frac{7}{5}(2^1-3^1(-1)^1),...,\frac{7}{5}(2^n-3^n(-1)^n),...) \\
    \frac{7x}{5}(\frac{1}{1-2x} - \frac{1}{1+3x}) \iff (0,\frac{7}{5}(2^0-3^0(-1)^0),\frac{7}{5}(2^1-3^1(-1)^1),...,\frac{7}{5}(2^n-3^n(-1)^n),...) 
\end{split}    
\end{equation}
Therefore we can say that this is the corresponding sequence: $\sum_{n=0}^{\infty}\frac{7}{5}(2^n-3^n(-1)^n)x^{n+1}$
\subsection*{d}
Firstly, we have to find $s_0$. $s_1=8s_0+10^{n-1}$ since $s_1=9$ we can find $s_0=1$. Let G(x) = $\sum_{k=0}^{\infty}a_nx^n$. Then from the given equation:
\begin{equation} 
\begin{split}
    \sum_{k=1}^{\infty}a_kx^k=\sum_{k=1}^{\infty}(8a_{k-1}+10^{k-1})x^k = 8x\sum_{k=1}^{\infty}a_{k-1}x^{k-1}+x\sum_{k=1}^{\infty}10^{k-1}x^{k-1} \implies \\
        G(x)-1=8xG(x)+x(1/1-10x) \implies G(x)=\frac{1-9x}{(1-8x)(1-10x)}
\end{split}    
\end{equation}
We can split this equation into two partial functions: $G(x) = \frac{1}{2}(\frac{1}{1-8x}+\frac{1}{1-10x})$\\
Also, we know that: \\
$\frac{1}{1-8x} \iff (8^0,8^1,8^2,8^3,8^4,8^5,....,8^n,...)$ \\
$\frac{1}{1-10x} \iff (10^0,10^1,10^2,10^3,10^4....,10^n,....)$ \\
Then using these two equations we construct these sequence:
\begin{equation} 
\begin{split}
   \frac{1}{1-8x}+\frac{1}{1-10x} \iff (8^0+10^0,8^1+10^1,....,8^n+10^n,....) \\
   \frac{1}{2}(\frac{1}{1-8x}+\frac{1}{1-10x}) \iff (\frac{8^0+10^0}{2},\frac{8^1+10^1}{2},....,\frac{8^n+10^n}{2},....) 
\end{split}    
\end{equation}
Therefore we can say that this is the corresponding sequence: 
$\sum_{n=0}^{\infty}\frac{1}{2}(8^n+10^n)x^n$
\section*{Answer 2}
\subsection*{a}
If $m|k$ then we can say that there exists a c $\in$ Z such that cm = k. To show $A_k\subseteq A_m$, assumed x $\in A_k$ since $A_k$ means set of numbers that are divisible by k, x = kt (t$\in$Z) and since k = cm we can say that x = tcm. Since $A_m$ means set of numbers that are divisible by m, x$\in A_m$ because x = tcm. Therefore $A_k\subseteq A_m$.
\subsection*{b}
By theorem (textbook pg.258), If n is a composite integer, then n has a prime divisor less than or equal to $\sqrt{n}$. Therefore, $C_n$'s biggest prime divisor is $\sqrt{n}$ and there is not a prime number bigger than $\sqrt{n}$, so this union operation is just a composite numbers union and these composite numbers are just multiplicity some constant with prime numbers which are biggest of them is $\sqrt{n}$. this is why right-hand side of equation equals it.
\subsection*{c}
$A_m$  means set of numbers that are divisible by m from (m,n]. Numbers that are divisible by m in this set is (2m,3m,...,n] the number of elements in this set is $\frac{n-2m}{m}+1$ we arrange this formula like $\frac{n}{m} -2 +1$ and it is equal to $\frac{n}{m}-1$ for $m \geq 2$ (it is in floor function because n may not be divisible by m). 
\subsection*{d}
Since ab is not included the interval of $A_{ab}$ this is the one number which is both a and b divides and not in the $A_{ab}$
\subsection*{e}
Since intersections of two prime is just product of these two numbers and their multiplicity with k (for instance a and b, intersections are ab,2ab,3ab...) it is $\lfloor n/ \prod_{x=p}^{p\in P} x \rfloor $
\subsection*{f}
By The Inclusion-Exclusion Princible, \\
$|C_{45}|=|A_2|+|A_3|+|A_5|-|A_2\cap A_3|-|A_3\cap A_5|-|A_2\cap A_5| + |A_2 \cap A_3 \cap A_5|$
\subsection*{g}
$|A_2| = 21$,$|A_3| = 14$,$|A_5| = 8$ and $|A_2\cap A_3| = 7$,$|A_3\cap A_5|=3$,$|A_2\cap A_5|=4$ and $|A_2 \cap A_3 \cap A_5| = 1$ then from the formula in part f: \\
21+14+8-7-3-4+1 = 30 this is the number of composite number to 45. And if we subtract this from 45 we can find number of primes: 45-30-1 = 14 (-1 because 1 is not prime).
\section*{Answer 3}
\subsection*{a}
A relation R on a set A is called transitive if whenever $(a, b) \in$R and $(b, c) \in$R, then $(a, c)  \in$R, for all $a,b,c\in$A. From this definition we should prove these cases from the given relation.\\ 1)Assume that $(a,b) << (c,d)$,$a<c$ and $(c,d) << (e,f)$,$c<e$ and then $(a,b) << (e,f)$  must be true since $a<c<e\implies a<e$ \\
2)Assume that $(a,b) << (a,d)$,$b\leq d$ and $(a,d) << (a,e)$,$d\leq e$ and then $(a,b) << (a,e)$ must be true since $b\leq d \leq e \implies b\leq e$ \\
3)Assume that $(a,b) << (c,d)$,$a<c$ and $(c,d) << (c,f)$,$d<f$ and then $(a,b) << (c,f)$  must be true since $a<c$ \\
4)Assume that $(a,b) << (a,d)$,$b\leq d$ and $(a,d) << (c,e)$,$a<c$ and then $(a,b) << (c,e)$ must be true since $a<c$ \\
Since for all cases transitivity is valid, This relation is transitive.
\subsection*{b}
A relation on a set A is called an equivalence relation if it is reflexive, symmetric, and
transitive. By the definition, we have to prove that $\propto$ is reflexive, symmetric, and
transitive.\\
1)\textbf{Reflexive}: Every f on the set of all functions from R to R f $\propto$ f since f(x) = f(x) for every $x\geq k$ 
2)\textbf{Symmetric}: If f $\propto$ g (means that (f,g) $\in \propto$) then we can easily say that g$\propto$ f (means that (g,f) $\in \propto$) since if f(x) = g(x)  for every $x\geq k$ then g(x) = f(x) for every $x\geq k$. \\
3)\textbf{Transitive}: If f $\propto$ g (means that (f,g) $\in \propto$) and g $\propto$ z (means that (g,z) $\in \propto$)  then we can easily say that f$\propto$ z (means that (f,z) $\in \propto$) since if f(x) = g(x) and g(x) = z(x)  for every $x\geq k$ then g(x) = z(x) since f(x)=g(x)=z(x) for every $x\geq k$. \\

\end{document}
